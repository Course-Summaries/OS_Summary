\documentclass[openany,12pt]{book}
\usepackage[utf8]{inputenc}
\usepackage{etoolbox}
\usepackage{hyperref}
\usepackage{geometry}
\usepackage{titlesec}
\usepackage{fancyhdr}

\usepackage{xr-hyper} % External referencing

\usepackage{longtable}

\usepackage{array}
\usepackage{booktabs}
\usepackage{listings}
\usepackage{colortbl}

\usepackage{enumitem}
\usepackage{amsthm}
\newtheorem*{definition}{Definition}

\newcommand{\code}[1]{\texttt{#1}}

\usepackage[dvipsnames, table]{xcolor}
\newcommand{\red}[1]{\textcolor{red}{#1}}
\newcommand{\blue}[1]{\textcolor{RoyalBlue}{#1}}
\newcommand{\gray}[1]{\textcolor{lightgray}{#1}}
\newcommand{\green}[1]{\textcolor{Green}{#1}}
\newcommand{\purple}[1]{\textcolor{Purple}{#1}}
\newcommand{\orange}[1]{\textcolor{BurntOrange}{#1}}




% =====================================================
% =====================================================
% =====================================================
% Define new counters
\newcounter{Topic}

% Format titles to look like chapters
\titleformat{\chapter}[display]
  {\normalfont\Huge\bfseries}{\chaptername\ \thechapter}{20pt}{\Huge}

\newcommand{\topicchaptername}{Topic}

% Custom "Topic Chapter"
\newcommand{\Topic}[1]{%
  \clearpage
  \refstepcounter{Topic}
  \renewcommand{\chaptername}{\topicchaptername}%
  \addcontentsline{toc}{chapter}{\topicchaptername\ \theTopic: #1}
  \chapter*{Topic \theTopic: #1}
  \markboth{Topic\theTopic}{#1}
}
% =====================================================
% =====================================================
% =====================================================




% =====================================================
% ============== Listings Formats =====================
% =====================================================

\geometry{margin=1in}
\titleformat{\section}{\large\bfseries}{\thesection}{1em}{}
\pagestyle{fancy}
\fancyhead[L]{Functions Reference}
\fancyhead[C]{Operating Systems 234123}
\fancyhead[R]{Spring 2025}
\setlength{\headheight}{15pt}

\title{Operating Systems (02340123)\\ Functions Reference - Spring 2025}
\author{Razi \& Yara}
\date{\today}

\newcommand{\functionEntry}[6]{%
  \noindent\rule{\linewidth}{0.5pt}
  \subsubsection*{\blue{\large{\texttt{#1}}}}
  \textbf{Declaration:} \texttt{#2} \\
  \textbf{Usage/Explanation:} #3 \\
  \textbf{Parameters:} #4 \\
  \textbf{Return Values:} #5 \\
  \textbf{Additional:} #6
  \vspace{1em}
}



\newcommand{\functionEntryPar}[6]{%
  \noindent\rule{\linewidth}{0.5pt}
  \subsubsection*{\blue{\large{\texttt{#1}}}}
  \textbf{Declaration:} \texttt{#2} \\
  \textbf{Usage/Explanation:} #3 \\
  \textbf{Parameters:}
  \begin{itemize}[leftmargin=*]
    #4
  \end{itemize}
  \textbf{Return Values:} #5 \\
  \textbf{Additional:} #6
  \vspace{1em}
}




\begin{document}

\maketitle
\tableofcontents
\newpage






\part{Functions Reference}



% =====================================================
% ======================= Topic 1 =====================
% =====================================================
\Topic{Process Management}

\functionEntry{fork}
{pid\_t fork();}
{Copies the parent process to the child process and returns with the two processes.

  Same code, Same memory, Same environment (files, etc.). But they are separate processes with separate memory spaces and different PIDs.}
{None}
{Returns 0 to child, PID of child to parent, -1 on error.}
{\label{func:fork}}



\functionEntry{wait}
{pid\_t wait(int *wstatus);}
{Waits until \textbf{any} child process ends, suspending the calling process until a child terminates}
{\code{wstatus}: Pointer to the variable where the exit status of the child will be stored. If NULL, no status is returned.

  To get the value of the status, you can use macros like \red{\code{WEXITSTATUS(*wstatus)}} which return the \red{second byte} of the variable, where the exit code of the child is stored.}
{If there are no children or all children have already terminated and waited for, it returns -1. Else waits until a child process ends and returns its PID.}
{Can only wait for direct child!\label{func:wait}}


\functionEntryPar{waitpid}
{pid\_t waitpid(pid\_t pid, int *wstatus, int options);}
{Wait until a specific child process ends.}
{\item \code{pid}: The PID of the child process to wait for. If -1, waits for any child process.
  \item \code{wstatus}: Pointer to an integer where the exit status of the child will be stored. If NULL, no status is returned.
  \item \code{options}: Options for waiting, such as WNOHANG (do not block if no child has exited). default is 0, which blocks until the child exits.}
{Returns the PID of the child that exited, or -1 on error. If WNOHANG is set and no child has exited, it returns 0.}
{\label{func:waitpid}}




\functionEntry{exit}
{void exit(int status);}
{Terminates the calling process and releases all of its recourses. The process becomes \textit{zombie} until its parent process requests to check its termination \gray{(e.g. wait())} and then clears completely.}
{\code{status}: The exit status of the process which is returned to the parent process when checked.}
{No return value. The process is terminated immediately, and \red{will never fail}.}
{The \code{main()} is not in fact the main function of the process, it is wrapped by \green{\code{int \_\_libc\_start\_main()}} who collects the return value of \code{main()} and calls \code{exit()} with it. This is why we don't use \code{exit()} in \code{main()} but rather return from it usually $\Longrightarrow$ \code{exit} is always called.\label{func:exit}}



\functionEntryPar{execv}
{int execv(const char *filename, char *const argv[]);}
{Replaces the current running process code with a new program. (same PID,PPID but different code and memory).}
{\item \code{filename}: The path to the file containing the program to execute.
  \item \code{argv}: An array of pointers to null-terminated strings containing the parameters to pass to the new program. The first element is the name of the process, i.e. \code{argv[0]=filename}. The last argument \red{must be \code{NULL}} to indicate the end of the array.}
{Returns -1 on error, and does not return on success as the current process is replaced by the new program.}
{The \code{execv()} function is one of the \textit{exec} family of functions, which replace the current process image with a new process image. \underline{It does not create a new process;} \underline{it replaces the current one}. The \code{v} stands for the array of arguments, \code{p} is for searching in the \code{PATH} environment variable for the filename. \label{func:execv}}



\functionEntry{getpid, getppid}
{pid\_t getpid();\\ pid\_t getppid();}
{\code{getpid()} returns the PID of the calling process, and \code{getppid()} returns the PID of the parent process \gray{(not real parent)}.}
{None}
{Returns the PID of the calling process or its parent accordingly.}
{\label{func:getpid}}








% =====================================================
% ======================= Topic 1 =====================
% =====================================================
\Topic{IPC \& I/O}

\functionEntryPar{kill}
{int kill(pid\_t pid, int sig);}
{sends signam num. \code{sig} to the process with PID \code{pid}.}
{\item \code{pid}: The PID of the process to send the signal to.\\
  If \code{pid} is 0, the signal is sent to all processes in the same process group as the calling process.
  \item \code{sig}: The signal number to send. \gray{(e.g. \code{SIGKILL}, \code{SIGTERM}, etc.)}.}
{Returns 0 on success, -1 on error. \gray{e.g. if the process doesn't exist}}
{Since there is no signal with the number 0, it is used to check if the process exists or not. If the process exists, it returns 0, else it returns -1. \gray{(e.g. \code{kill(<pid>,0)})}.\label{func:kill}}

\functionEntryPar{signal}
{sighandler\_t signal(int signum, sighandler\_t handler);\\
  \hspace*{10em}\gray{typedef sighandler\_t void (*sighandler\_t)(int);}}
{Sets a signal handler for the specified signal \code{signum}.}
{\item \code{signum}: The signal number to set the handler for. \gray{(e.g. \code{SIGINT}, \code{SIGTERM}, etc.)}.
  \item \code{handler}: The function to call when the signal is received. If \code{handler} is \code{SIG\_IGN}, the signal is ignored. If \code{handler} is \code{SIG\_DFL}, the default action for the signal is restored.}
{On success, returns the previous signal handler for the specified signal. If there was no previous handler, it returns \code{SIG\_DFL} or \code{SIG\_IGN}. On error, it returns \code{SIG\_ERR}.}
{\label{func:signal}}


\functionEntryPar{open}
{int open(const char *pathname, int flags, mode\_t mode); \gray{(mode\_t is optional)}}
{Opens the requested file by \code{pathname} for access with the properties specified by \code{flags} and permissions specified by \code{mode}.}
{\item \code{pathname}: The path to the file (or device) to open.
  \item \code{flags}: Flags that specify how the file should be opened. Must include one of the following:
  \begin{itemize}[leftmargin=*]
    \item \code{O\_RDONLY}: Open for reading only.
    \item \code{O\_WRONLY}: Open for writing only.
    \item \code{O\_RDWR}: Open for reading and writing.
  \end{itemize}
  Additional flags can be combined using the bitwise OR operator (\code{|}), such as:
  \begin{itemize}[leftmargin=*]
    \item \code{O\_CREAT}: Create the file if it does not exist.
    \item \code{O\_TRUNC}: Truncate the file to zero length if it already exists.
    \item \code{O\_APPEND}: Append data to the end of the file.
  \end{itemize}

  \item \code{mode}: Optional parameter that specifies the permissions of the file if it is created. It is used only if the \code{O\_CREAT} flag is set, and is a must!
  \gray{
    It is a bitwise OR of the following permission bits:
    \begin{itemize}[leftmargin=*]
      \item \code{S\_IRUSR}: Read permission for the owner.
      \item \code{S\_IWUSR}: Write permission for the owner.
      \item \code{S\_IXUSR}: Execute permission for the owner.
      \item \code{S\_IRGRP}: Read permission for the group.
      \item \code{S\_IWGRP}: Write permission for the group.
      \item \code{S\_IXGRP}: Execute permission for the group.
      \item \code{S\_IROTH}: Read permission for others.
      \item \code{S\_IWOTH}: Write permission for others.
      \item \code{S\_IXOTH}: Execute permission for others.
    \end{itemize}
  }
}
{In case of success, returns a file descriptor (an integer) that refers to the opened file. The given FD \blue{is the lowest available index in the FDT} of the process. If the file cannot be opened, it returns -1 and sets \code{errno} to indicate the error.}
{\label{func:open}}


\functionEntry{close}
{int close(int fd);}
{Closes the file descriptor \code{fd}, releasing the resources associated with it.}
{\code{fd}: The file descriptor to close. It must be a valid file descriptor that was previously opened using \code{open()} or similar functions.}
{Returns 0 on success, or -1 on error. If the file descriptor is invalid or already closed, it returns -1 and sets \code{errno} to indicate the error.}
{After closing an FD, you can no longer use it to access the file. \label{func:close}}


\functionEntryPar{read}
{ssize\_t read(int fd, void *buf, size\_t count);}
{Reads up to \code{count} bytes from the file descriptor \code{fd} into the buffer pointed to by \code{buf}.}
{\item \code{fd}: The file descriptor to read from.
  \item \code{buf}: A pointer to the buffer where the read data will be stored.
  \item \code{count}: The maximum number of bytes to read from the file descriptor.}
{In case of success, returns the number of bytes read (which can be less than \code{count} if fewer bytes are available). If the end of the file is reached, it returns 0. On error, it returns -1 and sets \code{errno} to indicate the error.}
{\code{read()} is a \red{blocking call} by default, meaning it will wait until data is available to read.\\
  Note that the \blue{seek pointer} of the \code{fd} is advanced by the number of bytes read, so subsequent reads will continue from where the last read left off.\label{func:read}}

\functionEntryPar{write}
{ssize\_t write(int fd, const void *buf, size\_t count);}
{Writes up to \code{count} bytes from the buffer pointed to by \code{buf} to the file descriptor \code{fd}.}
{\item \code{fd}: The file descriptor to write to.
  \item \code{buf}: A pointer to the buffer containing the data to write.
  \item \code{count}: The number of bytes to write from the buffer.}
{In case of success, returns the number of bytes written (which can be less than \code{count} if fewer bytes can be written). On error, it returns -1 and sets \code{errno} to indicate the error.}
{As with \code{read()}, \code{write()} is a \red{blocking call} by default, meaning it will wait until the data can be written. And the \blue{seek pointer} of the \code{fd} is advanced by the number of bytes written. \label{func:write}}



\functionEntryPar{pipe}
{int pipe(int filedes[2]);}
{Creates a unidirectional data channel \textit{(pipe)} with two FDs: one for reading and one for writing.}
{
  \item \code{filedes}: An array of two integers that the syscall will fill with:

  \begin{enumerate}[leftmargin=*]
    \item \blue{\code{filedes[0]}}: The file descriptor for \blue{reading} from the pipe.
    \item \blue{\code{filedes[1]}}: The file descriptor for \blue{writing} to the pipe.
  \end{enumerate}
}
{Return 0 if successful, otherwise -1.}
{\label{func:pipe}}



\functionEntryPar{dup, dup2}
{int dup(int oldfd);\\ int dup2(int oldfd, int newfd);}
{Creates a copy of the file descriptor \code{oldfd} and returns a new file descriptor that refers to the same open file description.}
{
  \item \code{oldfd}: The file descriptor to duplicate. Must be an open file descriptor.
  \item \code{newfd}: The desired new file descriptor (only for \code{dup2()}). If \code{newfd} is already open, it will be closed before being reused.
}
{
  For \code{dup()}: Returns the lowest numbered unused file descriptor.\\
  For \code{dup2()}: Returns \code{newfd} if successful, or -1 on error.
}
{If we want to close a file we need to close all of its FDs, which they all point to the same \textit{File Object}\label{func:dup}}


\functionEntryPar{mkfifo}
{int mkfifo(const char *pathname, mode\_t mode);}
{Creates a named pipe (FIFO) with the specified \code{pathname} and permissions \code{mode}.}
{
  \item \code{pathname}: The path where the FIFO will be created.
  \item \code{mode}: The permissions for the FIFO, similar to those used in the \code{open()} syscall. \blue{0777} means XRW for all users.
}
{0 on success, -1 on error.}
{FIFO are shown as files in the filesystem, but they are \red{not saved on disk}. It is a bidirectional communication channel between processes.\\
  \gray{Note: The FIFO must be opened by at least one process before any other process can write to it, i.e. it's blocking.}\\
  If opened for read \& write it will be \purple{non-blocking}\label{func:mkfifo}
}










\part{Cheat Sheet}


\begin{center}
  \textbf{Function Summary Cheat Sheet}
\end{center}
\vspace{-2em}

\renewcommand{\arraystretch}{1.2} % Adjust row height
\rowcolors{2}{gray!10}{white}
\begin{longtable}{|>{\raggedright\arraybackslash}p{3cm}|>{\raggedright\arraybackslash}p{12cm}|}
  \hline
  \rowcolor{RoyalBlue!25}
  \textbf{Function} & \textbf{Summary}                                                                                  \\
  \endfirsthead

  \hline
  \textbf{Function} & \textbf{Summary}                                                                                  \\
  \endhead

  \hline
  \code{fork()}     & Creates a child process identical to the parent. Returns 0 to child, PID to parent, -1 on error.  \\
  \code{wait()}     & Suspends the process until a child ends. Stores exit status in provided pointer.                  \\
  \code{waitpid()}  & Waits for a specific child to finish. Can use options like \code{WNOHANG}.                        \\
  \code{exit()}     & Terminates the calling process with a status code. Becomes a zombie until parent collects status. \\
  \code{execv()}    & Replaces the current process image with a new one using the provided filename and argument array. \\
  \code{getpid()}   & Returns the PID of the current process.                                                           \\
  \code{getppid()}  & Returns the PID of the parent process.                                                            \\
  \code{kill()}     & Sends a signal to a process by PID. Signal 0 used to check process existence.                     \\
  \code{signal()}   & Sets a signal handler for a given signal number.                                                  \\
  \code{open()}     & Opens a file with specified flags and permissions.                                                \\
  \code{close()}    & Closes a file descriptor.                                                                         \\
  \code{read()}     & Reads up to \code{count} bytes from a file descriptor into a buffer.                              \\
  \code{write()}    & Writes up to \code{count} bytes from a buffer to a file descriptor.                               \\
  \code{pipe()}     & Creates a unidirectional pipe with two FDs: one for reading, one for writing.                     \\
  \code{dup()}      & Duplicates an existing file descriptor, returns a new one.                                        \\
  \code{dup2()}     & Duplicates an FD to a specific new FD, closing the new FD first if open.                          \\
  \code{mkfifo()}   & Creates a named FIFO (pipe) with given pathname and mode.                                         \\
  \hline
\end{longtable}




\end{document}


