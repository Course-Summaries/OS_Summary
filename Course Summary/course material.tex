\documentclass[12pt]{article}
\usepackage[utf8]{inputenc}
\usepackage{hyperref}
\usepackage{geometry}
\usepackage{titlesec}
\usepackage{fancyhdr}
\usepackage{xr-hyper} % External referencing
\externaldocument[func:]{../Functions/functions} % Path to functions.aux

\geometry{margin=1in}
\titleformat{\section}{\large\bfseries}{\thesection}{1em}{}
\pagestyle{fancy}
\fancyhead[L]{Course Summary}
\fancyhead[C]{Operating Systems 234123}
\fancyhead[R]{Spring 2025}

\title{Operating Systems (02340123)\\ Summary}
\author{Razi \& Yara}
\date{\today}

\begin{document}

\maketitle
\tableofcontents
\newpage



% ====================================================
% ====================================================
% ====================================================
% ------------------- Notes Section ------------------
% ====================================================
% ====================================================
% ====================================================

\part{Lectures \& Tutorials}
\section{Processes and Signals}
A process is a schedulable entity executed by the OS. It has its own virtual memory space, stack, and heap. Key system calls include \texttt{fork()}\footnote{For more details, see \hyperref[func:fork]{Functions Reference}.} , \texttt{execv()}, and \texttt{wait()}.

Reference: \hyperref[func:fork]{\texttt{fork()}} (defined in the Functions file).

Sometimes, function details are important inline\footnote{Signals are asynchronous notifications sent to processes, handled in user mode.}:
\begin{quote}
    \textbf{Function:} \texttt{pid\_t fork(void);}
    \newline
    Usage: Creates a new process by duplicating the current one.
    \newline
    Return: 0 for child, child's PID for parent, -1 on failure.
\end{quote}

\section{Virtual Memory}
... (and so on for each lecture/topic)















% ==================================================== 
% ==================================================== 
% ==================================================== 
% ----------------- Summary Section ------------------
% ==================================================== 
% ==================================================== 
% ==================================================== 
\newpage
\part{Overall Summary}
Summarize core OS concepts: virtual memory, scheduling, synchronization, I/O, etc.

















% ==================================================== 
% ==================================================== 
% ==================================================== 
% --------------- Highlights Section ----------------=
% ==================================================== 
% ==================================================== 
% ==================================================== 
\newpage
\part{Highlights and Notes}
\begin{itemize}
    \item Processes are isolated, but related via parent-child hierarchy.
    \item \texttt{exec()} does not create a new process.
    \item Signals are async notifications - handled in user mode.
\end{itemize}




\end{document}